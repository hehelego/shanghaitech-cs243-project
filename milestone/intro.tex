\subsection{background}

The establishment of huge social network such as Tiktok, Wechat and Weibo has bring both challenges and opportunity.
Leveraging the connectivity of social network to promote social welfare has become a fruitful area of study in algorithmic game theory research , particularly since the publication of \cite{mechanism-design-in-social-network} in 2017.

Our project aims to develop a mechanism for facilitating general trading activity within social networks.

\subsection{prior work}

Initially, our research focused on distributed mechanisms that could be executed without prior knowledge of the network structure and without requiring a central authority to enforce the allocation.
In this regard, \cite{SRA} proposed the Sequential Resale Auction (SRA) mechanism for selling a single item with Ex-post Incentive Compatibility (EPIC) property.
Although SRA provides a sufficient solution for single-source single-item auction problems in distributed network settings,
realistic trading scenarios are often more complex than the settings of SRA.
For example, the network may contains multiple sellers who are willing to trade an item with potential buyers.
This requires a distributed double auction mechanism to be properly addressed.

However, after a brief literature review, we found that developing a double auction mechanism for social networks is a challenging task.
Even a centralized truthful mechanism for double auctions on general networks has yet to be developed.
As a result, we changed the direction of our research to focus on inventing a centralized mechanism.

\subsection{related work}

\cite{DNA} is the only paper that covers double auctions on a network. However, the mechanism proposed in that paper relies on unrealistic assumptions.
Specifically, it assumes that the graph of buyers can be decomposed into disjoint groups, which is unlikely to hold in modern social networks that are characterized by strong connectivity.
Therefore, we need to derive a mechanism that works for more general network structures.


\subsection{overall plan for research}

The overall plan for approaching the problem will involve conducting a literature review to establish connections between single seller mechanisms and our current multiple seller model, and using this information to develop a new mechanism that is suitable for our needs. The mechanism will then be subject to theoretical analysis using small test cases to verify key properties such as individual rationality (IR), incentive compatibility (IC), and weak budget balance (WB). Empirical evaluation will be conducted under different problem backgrounds in real life, using appropriate evaluation metrics to measure the mechanism's performance. Insights gained from the theoretical and empirical evaluations will be used to fine-tune and optimize the mechanism. Finally, the findings will be summarized and conclusions drawn about the mechanism's effectiveness, along with recommendations for future research in the field.
