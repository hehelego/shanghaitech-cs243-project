\documentclass[format=acmsmall, review=false, natbib=false]{acmart}
\usepackage{acm-ec-23}

\usepackage{biblatex} %Imports biblatex package
\addbibresource{ref.bib} %Import the bibliography file

\usepackage[ruled]{algorithm2e} % For algorithms
\renewcommand{\algorithmcfname}{ALGORITHM}
\SetAlFnt{\small}
\SetAlCapFnt{\small}
\SetAlCapNameFnt{\small}
\SetAlCapHSkip{0pt}
\IncMargin{-\parindent}

% Choose a citation style by commenting/uncommenting the appropriate line:

% Title. Note the optional short title for running heads. In the interest of anonymization, please do not include any acknowledgements.
\title{Project Proposal for Team$\#$6}

% Anonymized submission.
\author{Jintian Hu (2020533167), Huizhe Su (2020533009), Cheng Peng (2020533068), Liyu Yang (2020533162), Weiming Luo (2020533168)}

% Abstract. Note that this must come before \maketitle.

\begin{document}

% Title page for title and abstract only.
% \begin{titlepage}
\maketitle
% \end{titlepage}

% Paper body
\section{Topic: Generalized distributed auction on social networks}
Our research topic lies in the field of auction mechanism design on social networks.
A seller trys to sell one or more items through the network, and all the agents are potential buyers.
Each agent is connected to a subset of other agents, and, if the agent has participanted in the auction, 
it can invite its neighbors to the auction. With more agents, the seller will be more likely 
to sell the item with a higher price. However, agents are also competitors, so we have to design proper incentives
for them to inivte their neighbors.\par
Many centralized mechanisms have been proposed, but distributed mechanisms are still rare.
Our study will be based on Haoxin Liu's work, which lays the foundation for 
distributed mechanism design on social networks. Our work will be focused on generalizing Haoxin's work 
or exploring more properties of the proposed algorithms.

\section{Motivation}
% E.g., why is this topic/problem interesting?...

Previously designed mechanisms on social network 
require a centralized trustworthy executor to implement, exposing the privacy of agents to perilous situtaion.
To provide privacy security gurantee, the mechanism needs to be decentralized,
eliminating the need of trusted authority or a reliable thrid-party.

Research in decentralized mechansims is still in a primitive phase where various issues are not settled.

The first fully distributed mechanism \cite{liu2023distributed} has been proposed, however it is only solves single item auction.
Extension for general cases such as selling homogeneous items or combinatorial auction remains to be tackled.
Properties of the proposed mechanism are also not fully expolred yet.

Sybil-proofness \cite{stannat2021achieving} is another concern in social network-based mechanism design. 
It means avoiding buyers from creating fake nodes in social networks to improve their utilities,
which determines whether an mechanism can be effectively applied in reality.
Whether the first fully distributed mechanism is sybil-proofness,
and if not, how to design a sybil-proofness distributed mechanism,
are still questions to be answered.

\section{Research Questions}
% E.g., after exploring and reading papers, what is the problem that you will be investigating? Why is it interesting? How is your problem related to your reading? Are there any differences?
There are several appealing research directions:
\begin{enumerate}
    \item Current works only consider auctions with single item. It may be a simple extension that homogeneous items are sold with single-unit demand for each buyer. Models in which buyers with diminish valuation can also be considered.
    \item In reality, the participants are likely to suffer from some penalty when spreading the information and reselling the item, which brings about a diffusion cost. We may add this constraint to the existing distributed mechanism to see if we can reveal some interesting features. 
    \item Sybil attack is a common problem faced by most of the distributed system. We can evaluate the vulnerability to the newly proposed distributed mechanism and, if possible, give a sybil-proof modification.
\end{enumerate}

More specifically, the following  questions need to be addressed:
\begin{enumerate}
    \item How to design completely decentralized mechanisms for selling homogeneous items?
    \item What extension to the existing mechanisms can compensate the diffusion cost so that agents are willing to invite their neighbours?
    \item Whether or not an agent has an incentive to create fake identities? If it has, how to design a sybil-proof mechanism?
\end{enumerate}

\section{Plan}
% E.g., what method are you planning to use? What is the novelty?

The primary research methods for this study is literature review, mathematical modeling and theoretical analysis. 
We will first reveiw various literatures related to the above three directions,
which will allow us to choose one directions and define a more specific research question. 
By the end of week 9, we will write a comprehensive literature review on our selected topic. 
In the next stage, a formal model will be devised based on the review.
By the end of week 13, we will finish our milestone report where some preliminary solutions to the research questions will be proposed. 
We may carry out a simulation to verify it if an applicable result is derived before week 16.
The final report will be finished before the semester ends.



\nocite{*}
\printbibliography
\end{document}
