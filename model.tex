Consider a double auction market of $n+m$ participants
where $n$ out of them are sellers and the rest $m$ participants are potential buyers. Let $S=\{s_1,s_2,\ldots, s_n\}$ and $B=\{b_1,b_2,\ldots b_m\}$ be the set of all sellers and buyers.
Each seller $s_i\in S$ is holding a one-unit item and is willing to sell it at price no lower than $v^s_i$. Each buyer $b_j\in B$ wants to purchase one item with price no more than $v^b_j$.

Every participant of the market is connected to some other participants forming a social network. More precisely, every seller or buyer $x\in S\cup B$ can only communicate directly with its neighbors $r_x\subseteq (S\cup B\setminus \{x\})$.
Initially, only part of the buyers (called head buyers) are aware of the incoming auction.
Head buyers are those who is directly linked to a seller.
$H = \bigcup_{s_i \in S} (r_x \cap B)$ denotes all the head buyers in the market.


To sell the item at higher price, the sellers want to invite potential buyers other than the initial ones i.e, the head buyers, to expand the auction.
However, once neighboring potential buyers are invited to the auction, they will compete with the initial buyers for the items. Intuitively, a buyer has incentive to keep his neighboring buyers from entering the market.
Therefore, the market owner must adopt a mechanism encouraging the buyers in the market to inform the neighbors not being aware of the auction.

A mechanism for double auction on social network requires the buyers and the sellers to report their private information and decide the allocation of items and payment for each agent.
For the sellers, the mechanism asks them to reveal their expected price $\hat v^s=(\hat v^s_1,\hat v^s_2,\ldots \hat v^s_n)$.
The mechanism also requires the buyers to report their type $\hat\theta_i = (\hat v^b_i, \hat r_{s_i})$ and use $\hat\theta = (\hat\theta_1,\hat\theta_2,\ldots,\hat\theta_m)$ to denote a reported type profile.

After receiving the reported information $\hat\theta$ and $\hat v^s$, the mechanism determines the allocation $\pi^s,\pi^b$ and the payment $p^s,p^b$.
\begin{itemize}
	\item For each seller $s_i$:
		The allocation $\pi^s_i = 1$ means that he can trade the item with a buyer, otherwise $\pi^s_i = 0$.
		The payment $p^s_i\leq 0$ gives the money $s_i$ can get from the market owner.
	\item For each buyer $b_i$:
		The allocation $\pi^b_i = 1$ indicates that he get one item from a seller while $\pi^b_i=0$ if he get no item.
		The payment $p^b_i>0$ if $b_i$ is asked to pay money to the market owner
		and $p^b_i<0$ when $b_i$ can get get some rewards from the market owner.
\end{itemize}


We make a few definition of properties for further discussion.

\begin{definition}[active buyers]
	Given a profile $\hat\theta$, we say buyer $b_i$ is active
	if there exists $q_1,q_2,\ldots q_k$ such that
	$b_{q_1}\in H$ is a head buyer, $b_{q_k}=b_i$ is the active buyer and
	$b_{q_{i+1}} \in \hat r_{b_{q_i}}$ for every $i=1,2\ldots k-1$.
	That is a buyer is active if he is a head buyer or an active buyer invites him.\\
	Let $Q\subseteq B$ be the set of all active buyers.
\end{definition}

Next we definition the desired properties of the mechanism.

\begin{definition}[valid]
	A mechanism is valid if for all possible input of $(\hat\theta,\hat v^s)$ the following conditions are satisfied.
	\begin{itemize}
		\item If $b_i\not\in Q$ then $\pi^b_i=p^b_i=0$.
		\item $\sum_{b_i\in B} \pi^b_i = \sum_{s_i\in S} \pi^s$.
	\end{itemize}
\end{definition} 

\begin{definition}[utility]
	The utility is the welfare an agent gains from participanting the double auction mechanism trading.
	\begin{itemize}
		\item For $b_i\in B$, his utility is $u^b_i = \pi^b_i v^b_i-p^b_i$.
		\item For $s_i\in S$, his utility is $u^s_i = \pi^s_i v^s_i-p^s_i$.
	\end{itemize}
\end{definition}

\begin{definition}[individual rational (IR)]
	A mechanism is IR if for all possible input of $(\hat\theta,\hat v^s)$,
	\begin{itemize}
		\item For all $b_i\in B$, $u^b_i((\theta_i,\hat\theta_{-i}),\hat v^s)\geq 0$.
		\item For all $s_i\in S$, $u^s_i(\hat\theta,(v^s_i, \hat v^s_{-i})\geq 0$.
	\end{itemize}
	That is, no agent is punished for participating truthfully.
\end{definition}

\begin{definition}[incentive compatible (IC)]
	A mechanism is IC if
	\begin{itemize}
		\item For all $b_i\in B$,
			$u^b_i((\theta_i,\hat\theta_{-i}),\hat v^s)
			\geq u^b_i(\hat\theta,\hat v^s)$.
		\item For all $s_i\in S$,
			$u^s_i(\hat\theta,(v^s_i,\hat v^s_{-i}))
			\geq 
			u^s_i(\hat\theta,\hat v^s)$
	\end{itemize}
	That is, an agent's utility is maximized when he participate truthfully.
\end{definition}

\begin{definition}[weak budget balance (WBB)]
	A mechanism is WBB if
	$\sum_{s_i\in S} p^s_i(\hat\theta,\hat v^s)
	+\sum_{b_i\in B} p^b_i(\hat\theta,\hat v^s) \geq 0$
	for all $(\hat\theta,\hat v^s)$.
	That is, the market owner never pay extra money for running the mechanism.
\end{definition}
