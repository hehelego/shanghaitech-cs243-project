In recent years, research on double auctions has gained significant attention in the fields of economics and computer science. With the emergence of electronic markets, computer scientists are increasingly involved in the development of new market systems. Double auction markets can be viewed as centralized markets with specific trading rules where buyers and sellers conduct transactions according to regulations. A key issue in double auction research is to find a dominant strategy that allows sellers and buyers to genuinely report their respective valuations of merchandise.

In traditional double auction theory, we generally consider a fixed market consisting of buyers and sellers. However, in social networks, the absence of an effective double auction mechanism to facilitate the dissemination of information may lead to high valuation potential buyers missing out on auction information. Moreover, in traditional auctions, increasing profits for sellers and improving social welfare are usually two conflicting goals. By introducing the concept of social networking, we can motivate buyers who originally participated in the auction market to invite more potential buyers in the network to join, thus addressing both conflicting goals in traditional auctions simultaneously. It is crucial to design an auction mechanism that can achieve both advantages in social networks. However, the introduction of social networks means that we need to encourage buyers to not only report their merchandise valuations honestly but also disseminate auction information in social networks.

In this paper, we propose a novel scenario considering a social network with numerous interconnected sellers and buyers who conduct double auctions and incentivize buyers to invite neighbors to participate in the auction. Unlike existing research, our scenario focuses on intersecting buyer groups and the interconnections between buyers. This implies that a buyer may belong to multiple buyer groups simultaneously. Our goal is to find a double auction mechanism that motivates all buyers to invite other potential buyers in the social network while maintaining the advantages of traditional mechanisms, which cannot be achieved under existing mechanisms.

To address this issue, we first design a new double auction model in which buyers can interact with their neighbors in the social network and share auction information. In this model, we specifically focus on intersecting buyer groups and the mutual connections between buyers. Next, we propose and analyze a new auction mechanism that possesses incentive compatibility (IC) in the social network, meaning that honestly reporting true valuations is the optimal strategy for participants. We also study the performance of this mechanism in handling intersecting buyer groups, including aspects such as social welfare and budget balance.

To validate the effectiveness of our proposed auction mechanism, we prove its incentive compatibility through theoretical analysis. In addition, we conduct a series of simulation experiments to evaluate the performance of the mechanism under various conditions. The experimental results show that our auction mechanism achieves relatively high social welfare while maintaining budget balance when dealing with intersecting buyer groups, thereby addressing the two conflicting goals in traditional auctions to a certain extent.
